
\begin{table*}[!htbp]
\caption{Comparison of Reliability based Scheduling Algorithms}\label{T:h}
\centering
\scriptsize
\resizebox{\textwidth}{!}{%
\begin{tabular}{|p{1.9cm} p{2cm} p{2cm} p{2cm} p{2cm} p{2cm} p{2cm} p{2cm}|}
  \hline
  \textbf{Author} & \textbf{Basis} & \textbf{Advantages} & \textbf{Disadvantages} & \textbf{scheduling techniques} & \textbf{Experimental Environments} &\textbf{Performance matrices} 
 & \textbf{Taxonomy Classification} 
\\ \hline

Abdulhamid et al.~\cite{latiff2016fault} & Uncountable numeric nodes for resource in clouds & Lower makespan & No optimization & League championship algorithm & CloudSim & Failure ratio, the failure slowdown and the performance improvement rate & Compute -resource based, Task based
\\ \hline

Tang et al.~\cite{tang2016energy} &  Reliability and energy aware task scheduling architecture & To get good trade off among performance, reliability, and energy consumption & No support for cost optimization& Heuristic method & Discrete event simulation environment  & Schedule length, Energy consumption,  Application reliability & Energy based , energy source
 \\ \hline
 
 Jing et al.~\cite{jing2015reliability} &  Model for fault-tolerant aware scheduling &  Low complexity  & No cost, time optimization & Adaptive secure scheduling algorithm & Simulated environment & Reliability & Compute-resource based, Job based
\\ \hline


Malik et al.~\cite{malik2012reliability} & Reliability assessment mechanism for scheduling resources & Reliability assessment algorithms for general applications and real time applications  & No security and energy parameters consideration & Max -min & Amazon EC2 cloud & Fault tolerance, time & Compute resource based, Job based
\\ \hline



\end{tabular}
}
\end{table*}