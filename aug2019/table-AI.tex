\begin{table*}[!htbp]
\caption{Comparison of AI based Scheduling Algorithms}\label{T:j1}
\centering
\scriptsize
\resizebox{\textwidth}{!}{%
\begin{tabular}{|p{1.9cm} p{2cm} p{2cm} p{2cm} p{2cm} p{2cm} p{2cm} p{2cm}|}
  \hline
  % after \\: \hline or \cline{col1-col2} \cline{col3-col4} ...
\textbf{Author} & \textbf{Basis} & \textbf{Advantages} & \textbf{Disadvantages} & \textbf{scheduling techniques} & \textbf{Experimental Scale} &\textbf{Experimental Parameters}
& \textbf{Taxonomy Classification} 
\\ \hline

Mingxi Cheng et al.~\cite{cheng2018drl} &  Two stage resource provisioning and/or task scheduling processor & Consideration of energy consumption aspects & No security consideration  & Reinforcement approach & Simulation Testbed & Energy cost and runtime & compute resource based, job based
\\ \hline


Hongzi Mao et al. ~\cite{mao2018learning} & Presented Decima scheduler & Consideration of low latency scheduling decisions & High complexity & Reinforcement learning method & Simulated environment  & Execution time & Compute resource based, workflow based
\\ \hline

Qingchen Zhang et al. ~\cite{zhang2017energy} & A hybrid dynamic voltage and frequency scaling (DVFS) scheduling algorithm   & Minimize energy consumption & does not handle security & Deep Q-learning model & Simulation environment & energy consumption & compute resource based, job based
\\ \hline

Yuandou Wang et al. ~\cite{wang2019multi} & Focused on workflow scheduling & Minimize energy consumption & Ignored time, security & Markov method & Simulated environment & Energy efficiency & Compute resource based, workflow
\\ \hline

Enda Barret et al. ~\cite{barrett2013applying} & Guided  an optimal decision in the process of resource allocation & Reduced Time & Ignored security & Deep Q learning method & Simulation & Makespan & Compute resource based, task based
\\ \hline


\end{tabular}
}

\end{table*}
