%\begin{sidewaystable*}[!htbp]
\begin{table}[htb]
\caption{Comparison of ML-based Scheduling Algorithms}\label{T:j1}
 \hspace{8pt}
\centering
\scriptsize
\resizebox{\textwidth}{!}{%
\begin{tabular}{|g{2cm} p{2cm} p{2cm} p{2cm} p{2cm} p{2cm} p{2cm} p{2cm}|}
  \hline
  % after \\: \hline or \cline{col1-col2} \cline{col3-col4} ...
 \textbf{Classification} & \textbf{Author} & \textbf{Basis} & \textbf{Advantages} & \textbf{Disadvantages} & \textbf{scheduling techniques} & \textbf{Experimental Scale} &\textbf{Experimental Parameters}


\\ \hline

\makecell{Compute \\resource-based,\\ Job-based,\\ ML-based} & Mingxi Cheng et al.~\cite{cheng2018drl} &  Two stage resource provisioning and/or task scheduling processor & Consideration of energy consumption aspects & No security consideration  & Reinforcement approach & Simulation Testbed & Energy cost and runtime
\\ \hline


\makecell{Compute \\resource-based,\\ Workflow-based,\\ ML-based} & Hongzi Mao et al. ~\cite{mao2018learning} & Presented Decima scheduler & Consideration of low latency scheduling decisions & High complexity & Reinforcement learning method & Simulated environment  & Execution time
\\ \hline

\makecell{Compute \\resource-based,\\ Job-based,\\ ML-based} & Qingchen Zhang et al. ~\cite{zhang2017energy} & A hybrid dynamic voltage and frequency scaling (DVFS) scheduling algorithm   & Minimize energy consumption & does not handle security & Deep Q-learning model & Simulation environment & energy consumption
\\ \hline

\makecell{Compute \\resource-based,\\ Workflow-based,\\ ML-based} & Yuandou Wang et al. ~\cite{wang2019multi} & Focused on workflow scheduling & Minimize energy consumption & Ignored time, security & Markov method & Simulated environment & Energy efficiency
\\ \hline

\makecell{Compute \\resource-based,\\ Task-based,\\ ML-based} & Enda Barret et al. ~\cite{barrett2013applying} & Guided  an optimal decision in the process of resource allocation & Reduced Time & Ignored security & Deep Q learning method & Simulation & Makespan
\\ \hline


\end{tabular}
}
\end{table}
%\end{sidewaystable*}
