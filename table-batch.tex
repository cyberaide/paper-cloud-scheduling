\begin{sidewaystable*}[p]

  % \centering
\caption{Comparison of Different Batch Resource Management Systems}\label{T:l}
\hspace{8pt}
\centering
\scriptsize
\resizebox{\textwidth}{!}{%
\begin{tabular}{|g{2cm} p{1.9cm} p{2.3cm} p{2cm} p{1.5cm} p{2cm}|}
  \hline
 \textbf{Classifaction} &   
 \textbf{Framework} & \textbf{Batch Features} &  \textbf{Cloud Bursting} & \textbf{Containers} & \textbf{Comment}  
\\ \hline

\makecell{Cloud bursting,\\ Cluster model} & Slurm~\cite{www-slurm} & 
Policy driven, backfill, exclusive and non-exclusive access to compute nodes & 
AWS, Azure, Google, Oracle & 
Yes & 
Open Source, popular
  \\ \hline

\makecell{Cloud bursting,\\ Cluster model} & Univa Grid Engine~\cite{www-univa} & 
Policy driven, backfill, exclusive and non-exclusive access to compute nodes, fault tolerant master & 
AWS, Azure, Google & 
Yes & 
previously SUN Grid Engine, Genias Codine
\\ \hline  
  
\makecell{Cloud bursting,\\ Cluster model} & Load Sharing Facility (LSF)~\cite{www-lsf} & 
Policy driven, backfill, exclusive, non-exclusive access to compute nodes &  IBM Cloud, AWS, Google and Azure & 
Yes & 
Previously OpenLava, IBM  Open Source 
\\ \hline

\makecell{Cloud bursting,\\ Cluster model} & Moab~\cite{www-moab} & 
Fairness policies, dynamic priorities, and extensive reservations &
AWS, Azure, Oracle, Google &
Yes & 
Open Source
\\ \hline
  
\makecell{Cloud bursting,\\ Cluster model} & Open Portable Batch System (OpenPBS)~\cite{Openpbs2018,Henderson1995} & 
Policy driven, backfill, exclusive and non-exclusive access to compute nodes, fault tolerant master & 
AWS, Azure, Google, Oracle & 
Yes & 
Open Source
\\ \hline



\end{tabular}
}

\end{sidewaystable*}
