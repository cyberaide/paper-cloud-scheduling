\RequirePackage[hyphens]{url}
%\documentclass[10pt,twocolumn]{article}
%\documentclass[preprint,12pt,3p]{elsarticle}

\documentclass[final,5p,times,twocolumn]{elsarticle}

\usepackage{pdflscape}
\usepackage[tablesonly]{endfloat}
\usepackage{fancyhdr}



\usepackage[T1]{fontenc}
\usepackage{comment}
\usepackage{fullpage}
\usepackage{graphicx}
\usepackage{adjustbox}
\usepackage{todonotes}
\usepackage{multirow}
\usepackage{makecell}
\usepackage{xcolor}
\usepackage{tikz}
\usetikzlibrary{arrows}
\usetikzlibrary{mindmap,trees}
\usetikzlibrary{backgrounds,shapes,arrows,positioning,calc,snakes,fit}
\usepgflibrary{decorations.markings}
\usepackage{color, colortbl}
\usepackage[T1]{fontenc}
%\usepackage[table]{xcolor}
\definecolor{Gray}{gray}{0.9}
\usepackage{forest}
\usepackage{hyperref}
\usepackage{adjustbox}
\usepackage[figuresright]{rotating}
 
\usetikzlibrary{arrows.meta,shadows}

\newcommand{\ngreen}{bottom color=green!20}
\newcommand{\ngrey}{bottom color=gray!20}
\newcommand{\nred}{bottom color=red!20}
\newcommand{\nwhite}{bottom color=white!20}
\newcommand{\TODO}[1]{\todo[inline]{#1}}
\newcommand{\GVL}[1]{{\begin{blue} #1}}

%\usepackage[linguistics]{forest}
\usepackage{smartdiagram}

\setcounter{secnumdepth}{6}
\setcounter{tocdepth}{6}

\forestset{
  skan tree/.style={
    for tree={
      drop shadow,
      text width=3cm,
      grow'=0,
      rounded corners,
      draw,
      top color=white,
      bottom color=blue!20,
      edge={Latex-},
      child anchor=parent,
      %parent anchor=children,
      anchor=parent,
      tier/.wrap pgfmath arg={tier ##1}{level()},
      s sep+=2.5pt,
      l sep+=2.5pt,
      edge path'={
        (.child anchor) -- ++(-10pt,0) -- (!u.parent anchor)
      },
      node options={ align=center },
    },
    before typesetting nodes={
      for tree={
        content/.wrap value={\strut ##1},
      },
    },
  },
}

\newcommand{\TITLE}{Cloud Resource Scheduling Taxonomy }

\author[label1]{Gregor von Laszewski\corref{cor1}\fnref{label3}}
\address[iu]{{\small $^1$ Intelligent Systems Engineering Dep., Indiana University, Bloomington, IN 47408, USA.}
}
\cortext[cor1]{Corresponding author}
\ead{laszewski@gmail.com}
\ead[url]{https://laszewski.github.io/}

\author[label2]{Rajni Aron}
\address[punjab]{School of Computer Science and Engineering, Lovely Professional University, Punjab, India}


\author[label1]{Geoffrey C. Fox}



% in final version remove the following line, it is included
% so we can review better
\usepackage[nomarkers,tablesonly]{endfloat}

%\makeatletter
%\def\ps@pprintTitle{%
%   \let\@oddhead\@empty
%   \let\@evenhead\@empty
%   \let\@oddfoot\@empty
%   \let\@evenfoot\@oddfoot
%}
%\makeatother

\makeatletter
\newenvironment{rotatepage}
        {%
            \if@twoside%
                \ifthispageodd{\pagebreak[4]\global\pdfpageattr\expandafter{\the\pdfpageattr/Rotate 90}}{%
                \pagebreak[4]\global\pdfpageattr\expandafter{\the\pdfpageattr/Rotate 270}}%
            \else%
                \pagebreak[4]\global\pdfpageattr\expandafter{\the\pdfpageattr/Rotate 90}%
            \fi%
        }%
        {\pagebreak[4]\global\pdfpageattr\expandafter{\the\pdfpageattr/Rotate 0}}%
\makeatother

\BeforeBeginEnvironment{sidewaystable}{\begin{rotatepage}}
  \AfterEndEnvironment{sidewaystable}{\end{rotatepage}}

\begin{document}

\onecolumn

%\parindent0pt Draft:\\

%{\Large \TITLE\\}

%\setcounter{tocdepth}{3}
%\tableofcontents
%\newpage

%\twocolumn



\begin{frontmatter}
\title{\TITLE}

\maketitle



\begin{keyword}

  Y-Cloud Taxonomy,
  Cloud Scheduling,
  Scheduling Virtual Machines,
  Scheduling Containers,
  Scheduling FaaS

\end{keyword}

\begin{abstract}

  The growth and development of scientific applications in the cloud
  demands the creation of efficient resource management systems. Due
  to the scale of resources, the heterogeneity of services, their
  inter-dependencies and unpredictability of load this is a complex
  problem. We present a resource scheduling taxonomy that originates
  from the experience in utilizing and managing multi-cloud
  environments.  This study is backed up by a literature review that
  targets not only virtual machine, but also container and Function as
  a Service frameworks. It justifies a proposed resource provider
  focused Y-cloud taxonomy and provides a an overview of existing
  scheduling techniques in cloud computing.  As a result this work can
  lead to a better understanding of the complex field of scheduling
  for clouds in general. Furthermore, the study promotes through the
  Y-cloud taxonomy the vision of a layered scheduling architecture
  that will be useful for the implementation of application and
  resource-based scheduling frameworks in support of the NIST Big Data
  Reference Architecture.

\end{abstract}

\end{frontmatter}

\section{Tables}


\newcolumntype{g}{>{\columncolor{Gray}}p}
\begin{table*}[!htbp]
 \caption{Comparison of Dynamic Scheduling Algorithms}
     \label{T:dynamic-scheduling}
   \centering
\scriptsize
\resizebox{\textwidth}{!}{%
\begin{tabular}{|g{2cm} p{2cm} p{2cm} p{2cm} p{2cm} p{2cm} p{2cm} p{1.5cm}|}
  \hline
 \textbf{Taxonomy Classification} & \textbf{Author} & \textbf{Basis} & \textbf{Advantages} & \textbf{Disadvantages} & \textbf{scheduling techniques} & \textbf{Experimental Scale} &\textbf{Experimental Parameters}

\\ \hline

 Energy-basedd, energy source & Sun et al.~\cite{sun2015live} & Introduction of a virtual data center to solve VM migration issues & Low complexity & Fixed band-with & Heuristic algorithm & Simulated environment & VM migration cost and time
\\ \hline

Energy-basedd, VM migration & Tighe et al.~\cite{tighe2014integrating} & Auto scaling  algorithm  alongside  a  dynamic VM  allocation  algorithm  & Reduce a number of migrations & No optimization criteria &Rule based heuristic&  DCSim & Power, SLA, Migrations 
\\ \hline

Energy-based & Keller et al.~\cite{keller2014hierarchical} &  Management of data center resources to reduce the management scope &  Reducing  the  overhead  in  the  data centre management network & High complexity & Greedy algorithm & DCSim  & Power, number of migrations, average number of racks, active hosts 
\\ \hline 

Energy-based, PUE & Tighe et al.~\cite{tighe2013distributed} &  Trade-off  among number of  migrations, SLA  violation and power & Consider energy and SLA & more bandwidth usage & First fit algorithm &  DCSim  & Power consumption, number of migrations, SLA violations 
\\ \hline 

Compute-resource based, VM based & Tordsson et al.~\cite{tordsson2012cloud} &  Optimized placement of applications in multi-cloud environments & Emphasized on Price and performance in terms of hardware configuration, load balancing & Ignored security and energy efficiency at time of scheduling & Integer programming formulations & Amazon EC2 & Throughput, number of jobs 
\\ \hline


Energy-based, VM-based & Younge et al.~\cite{las10cloudsched} & 
Power aware & 
reduces cost & 
slight reduction in performance & 
heuristic & on premise cloud & 
power consumption 

\\ \hline

\end{tabular}
}
\end{table*}

\begin{sidewaystable*}[!htbp]

\caption{Comparison of Energy-aware VM Placement-based Scheduling
  Algorithms (A)}\label{T:g-a}
 \hspace{8pt}
\centering
\scriptsize
\resizebox{\textwidth}{!}{%
\begin{tabular}{|g{2.1cm} p{1.5cm} p{2cm} p{2cm} p{2cm} p{2cm} p{2cm} p{2cm}|}
  \hline
  % after \\: \hline or \cline{col1-col2} \cline{col3-col4} ...
\textbf{Classification} & \textbf{Author} & \textbf{Basis} & \textbf{Advantages} & \textbf{Disadvantages} & \textbf{scheduling techniques} & \textbf{Experimental Scale} &\textbf{Experimental Parameters} 
\\ \hline

  \makecell{Energy source, \\Compute\\ resource-based, \\Job-based} & Calheiros et al.~\cite{calheiros2014energy} & Intelligent scheduling combined DVFS capability  &  Improved energy efficiency  & Ignored Network and Storage energy consumption  &  Rank method   & Cloud Sim & Energy consumption
\\ \hline 

\makecell{Energy-aware, \\PUE} & Bessis et al.~\cite{bessis2013using} & Improving communication for Distributed systems at the time of scheduling & Improved system performance & High complexity & Graph theory concepts & SIMIC & makespan, latency times 
\\ \hline 
  
\makecell{Energy-aware,\\ Energy source, \\Cost, \\VM-based} & Bi et al.~\cite{bi2017application} & Dynamic Scheduling algorithm for reducing energy consumption & Focused on performance and energy cost & High complexity due to virtualized Data centers & meta heuristic methods & Simulated environment & Profit, CPU utilization  
\\ \hline   

    \makecell{Energy-aware,\\ Compute\\ resource-based, \\VM-based} & Duan et al.~\cite{duan2016energy} & Scheduling of VM machines & Improved the CPU load prediction & No optimization & Ant colony optimization &  CloudSim & Energy Consumption
\\  \hline 

  \makecell{Energy-aware,\\ Energy source,\\ Cost, \\VM-based} & Bui  et al.~\cite{bui2017energy} & Balance between energy efficiency and quality of service & Low complexity & Ignored cost, scalability & Greedy first fit algorithm & Simulated Environment & Energy, Memory, CPU  
\\ \hline


\makecell{Energy-aware,\\ Energy source, \\ Cost,\\ VM-based} & Zhu et al.~\cite{zhu2017three} & Data center balance while saving power consumption & Improved management of VM resource & High complexity &  Multi-dimensional vector bin packing problem-based heuristic & CloudSim & SLA violations, Resource utilization
\\ \hline

\makecell{Energy-aware, \\ Energy source,\\ VM-based} & Beloglazov et al.~\cite{beloglazov2010energy} &  Enhancement of resource utilization by re-allocation of the resources. & Considered different types of workloads, No prior information about applications & Ignored cost and time & Heuristic algorithm & CloudSim & Energy, Average SLA, migrations
  \\ \hline

  
\makecell{Energy source,\\ Compute\\ resource-based , \\VM-based}  & Quarati et al.~\cite{quarati2013hybrid} &  Reservation of a quota of private resources & Reduced energy consumption and carbon emission & Lacks implementation on a real-world cloud platform & Round robin algorithm & Discrete Event Simulator & User satisfaction, energy saving, energy consumption
\\ \hline

  \makecell{PUE, \\ Job-based} & Garg et al.~\cite{garg2011environment} &  Optimal scheduling policies & Reduced energy cost, energy consumption & Ignored security & Meta-scheduling policies & Simulated environment  & Average energy consumption, average carbon emission, arrival rate of application
 \\ \hline
 

\end{tabular}
}
\end{sidewaystable*}
  

 \begin{sidewaystable*}[!htbp]

   \caption{Comparison of Energy-aware VM Placement-based Scheduling Algorithms (B)}\label{T:g-b}
   \hspace{8pt}
  
\centering
\scriptsize
\resizebox{\textwidth}{!}{%
\begin{tabular}{|g{2cm} p{1.5cm} p{2cm} p{2cm} p{2cm} p{2cm} p{2cm} p{2cm}|}
  \hline
  % after \\: \hline or \cline{col1-col2} \cline{col3-col4} ...
\textbf{Classification} & \textbf{Author} & \textbf{Basis} & \textbf{Advantages} & \textbf{Disadvantages} & \textbf{scheduling techniques} & \textbf{Experimental Scale} &\textbf{Experimental Parameters} 
   
\\ \hline

\makecell{Energy-aware,\\ VM-based} & Keke et al.~\cite{gai2016dynamic} & Cloudlets for energy reduction & Reduced energy consumption & No time consideration & FCFS scheduling policy & DECM-Sim & Energy consumption 
\\ \hline
  
\makecell{Energy source,\\ VM-based} & Dabbagh et al.~\cite{dabbagh2015energy} &  Energy-aware resource management decisions  & improved performance & No optimization criteria, high complexity& K-means clustering& Testbed & Average CPU and Network utilization
\\ \hline

    \makecell{Energy source, \\VM-based} & Kim et al.~\cite{kim2014energy} & VM energy consumption estimation model & Reduced cost, power consumption & More complex to implement, Ignored time & Power aware scheduling algorithm & Xen 4.0 hypervisor & Energy consumption, error rate
\\ \hline

\makecell{Compute\\ resource-based,\\ VM-based} & Van Do et al. in~\cite{van2012comparison} & Interaction aspects between on-demand requests and the allocation of virtual machines & Reduced energy consumption & No cost and time optimization & Power aware scheduling algorithm & Numerical Simulation & Average Energy consumption, average heat emission
\\ \hline

  
  \makecell{Energy-aware,\\ Energy source, \\Compute \\resource-based, \\
  VM-based (migration),\\ Workflow} & Li et al.~\cite{li2016energy} & Scheduling algorithm to reduce energy consumption while meeting the deadline constraint & Focused on energy consumption & Ignored  processing power energy consumption, VM migration & Heuristic method & Simulated environment & Energy consumption
\\ \hline 


  \makecell{Energy source,\\PUE, \\ VM-based} & Ding  et al.~\cite{ding2015energy} & Dynamic VMs scheduling & Increased Processing Capacity & Ignored VM migration, Power penalties of status transitions of processor & FCFS & Simulated environment  & Deadline, Energy consumption 
\\ \hline 



\end{tabular}
}
\end{sidewaystable*}

  \begin{sidewaystable*}[p]
\caption{Comparison of Network-aware VM Placement-based Scheduling Algorithms}\label{T:c}
\centering
\scriptsize
\resizebox{\textwidth}{!}{%
\begin{tabular}{|g{2cm} p{2cm} p{2cm} p{2cm} p{2cm} p{2cm} p{2cm} p{1.9cm}|}
  \hline
   \textbf{Classification} & \textbf{Author} & \textbf{Basis} & \textbf{Advantages} & \textbf{Disadvantages} & \textbf{scheduling techniques} & \textbf{Experimental Scale} &\textbf{Experimental Parameters} 
   
\\ \hline

VM based & Yu et al.~\cite{yu2017survivable} & Service provisioning on IaaS platform while focusing on the inter-connected
VMs. &  High availability & High complexity & Heuristic algorithm & Simulator & Average VM consumption ratio, average running time 
\\ \hline 

Compute resource based, VM based & Bi et al.~\cite{bi2015sla}~\cite{bi2016trs} & Architecture for self management of data centers & Considered temporal request of multi-tier web applications  & does not consider security parameters  & Queuing approach &  trace-driven simulation & Cost 
\\ \hline

 Compute resource based & Rampersaud et al.~\cite{rampersaud2016sharing} & Used page-sharing concept to handle VM Packing problem & Improvement of memory sharing during allocation decisions & High complexity & Linear programming technique & Simulated environment & Memory reduction, number of excess servers 
\\ \hline

 Compute resource-VM based & Lucrezia et al.~\cite{lucrezia2015introducing} & Investigated OpenStack for the deployment of network service graphs & Increased throughput & Analyzing time is more, Ignored policy-constraints in  order  to  define  administration  rules&  Brute force algorithm  & KVM hypervisors &  VM locations, traffic throughput and latency 
\\ \hline 

Compute resource VM based & Biran et al.~\cite{biran2012stable}& Consideration of traffic  bursts  in  deployed  services & Minimizing  the  maximum load  ratio  over  all  the  network & Ignored energy consumption& Greedy heuristic algorithm & Testbed & Average packet delivery delay , placement solving time  
\\ \hline




Compute-resource based workflow based & Kondikoppa et al.~\cite {kondikoppa2012network} & To make Hadoop scheduler aware of network topology & Improved data locality& Ignored cost, energy, security & FIFO &Eucalyptus  based testbed & Execution time, delay for scheduling task 


\\ \hline

\end{tabular}
}
\end{sidewaystable*}

%\begin{sidewaystable*}[p]
\begin{table}[htb]
\caption{Comparison of Cost-based Scheduling Algorithms}\label{T:e}
\hspace{8pt}
\centering
\scriptsize
\resizebox{\textwidth}{!}{%
\begin{tabular}{|g{2cm} p{2cm} p{2cm} p{2cm} p{2cm} p{2cm} p{2cm} p{1.9cm}|}
  \hline
   \textbf{Classification} & \textbf{Author} & \textbf{Basis} & \textbf{Advantages} & \textbf{Disadvantages} & \textbf{scheduling techniques} & \textbf{Experimental Scale} &\textbf{Experimental Parameters} 

\\ \hline                                                                                                                                                                              
                                                                                                                                                                              
  \makecell{Compute\\ resource-based,\\ Task-based,\\ Cost-based} & Bi et al.~\cite{bi2015sla}~\cite{bi2016trs} & Architecture for self management of data centers & Considered temporal request of multi-tier web applications  & does not consider security parameters  & Queuing approach &  trace-driven simulation & Cost
  \\ \hline
  
\makecell{Compute\\ resource-based,\\  data-based,\\  task,\\  Latency,\\  VM-based,\\ Cost-based} & Yuan et al.~\cite{yuan2017ttsa,yuan2017temporal} & Emphasizing profit maximization & handles service delay bound & High complexity  & PSO and SA & simulation environment & Revenue
\\ \hline

\makecell{Compute\\ resource-based,\\  Task-based,\\ Cost-based}  & Zuo et al.~\cite{zuo2015multi} & Multi-objective Task Scheduling  & Improved  performance & Ignored energy consumption & Ant colony optimization & CloudSim & Cost, makespan, deadline violation rate
\\ \hline

\makecell{Compute\\ resource-based,\\  Workflow-based,\\ Cost-based} & Arabnejad et al.~\cite{arabnejad2015cost} &  Re-use of pre-provisioned instances for scheduling & Less complexity& Ignored security and energy efficiency& Deadline early Tree algorithm & CloudSim & Cost and deadline
\\ \hline

\makecell{Compute\\ resource-based,\\ VM-based,\\ Cost-based} & Wu et al.~\cite{wu2012sla} &  VM usage efficiency designed utility function by considering dynamic VM deploying time, processing time and data transfer time & Improved cost saving & Does not support security and energy efficient  & Admission control and scheduling algorithm & CloudSim & Average response time, total profit
\\ \hline

\makecell{Data-based,\\ Cost-based} & Lee et al.~\cite{lee2012profit} &  Personalized
features of the user request and the elasticity of SLA properties & Reduced operational costs and increase profits & Objectives conflict with each other & binary integer programming & CloudSim & Average utilization, average net profit rate, average response time  
  \\ \hline


\makecell{Compute\\ resource-based,\\  VM-based,\\ Cost-based} & Ari et al.~\cite{ari2013design} &  Finite Element Analysis cloud service with a focus on mechanical structural analysis, performance characterization and job scheduling issues & Throughput improvement and resource utilization & Ignored cost& Adaptive algorithm & Testbed & Throughput and time
\\ \hline

  
\end{tabular}
}
\end{table}
%\end{sidewaystable*}

\begin{sidewaystable*}[!htbp]
   \caption{Comparison of Time-based Scheduling Algorithms}
    \label{T:f}
\centering
\scriptsize
\resizebox{\textwidth}{!}{%
\begin{tabular}{|g{2cm} p{1.9cm} p{2cm} p{2cm} p{2cm} p{2cm} p{2cm} p{2cm}|}
  \hline
  % after \\: \hline or \cline{col1-col2} \cline{col3-col4} ...
 \textbf{Classification} & \textbf{Author} & \textbf{Basis} & \textbf{Advantages} & \textbf{Disadvantages} & \textbf{scheduling techniques} & \textbf{Experimental Scale} & \textbf{Experimental Parameters} 
 
\\ \hline

Compute-resource based, task & Yuan et al.~\cite{yuan2017time} & Task scheduling in green data centers & Investigated temporal variations &  Ignored energy consumption and cost & PSO and SA & Simulated Environment & Delay bound and time 
\\ \hline

 Compute resource based workflow & Arabnejad et al.~\cite{arabnejad2017scheduling} & Dynamically provisioned commercial cloud environments & Evaluation of task selection algorithms reveals impact of workflow symmetry & High complexity & Rank method & CloudSim & Response time, Cost 
\\ \hline

Compute-resource based, Workflow & Thomas et al.~\cite{thomas2015credit} & Task length aware scheduling & Lesser makespan and increased resource utilization & No comparison with existing algorithm & Min-min & CloudSim & Makespan 
\\ \hline 

Compute-resource based, VM & Frincu~\cite{frincu2014scheduling} & A priory scheduling and  searching for an optimal allocation of components on nodes in order to ensure a homogeneous spread of component types on every node. & Minimizing the application cost & Centralized approach represents a single point of failure & Nonlinear-programming & Simulator platform & Average load per node, optimal allocation, reliability 
\\ \hline

Compute-resource based, VM & Erdil~\cite{erdil2013autonomic} & Disseminated information as agents of dissemination sources for resource scheduling & Availability of resource state, reduces dissemination overhead & Ignored cost as parameters & Adaptive proxy algorithm & Scalable simulation network framework & Query satisfaction rates, random walk hop count limit 
\\ \hline


 Compute-resource based, Data based, Task-based & Van den Bossche et al. in~\cite{vandenbosshe2013} & Deadline-based workloads in a hybrid cloud setting & Minimize cost and time & does not handle multiple workflows & hybrid scheduling approach & Simulator & Total Cost, application deadline met, turnaround time, data transferred 
\\ \hline

Compute-resource based , task & Xu et al.~\cite{xu2011job} & Berger model and assign tasks on optimal resources to meet user's QoS requirements & Optimal completion time & Ignored cost and energy efficiency, security& Resource allocation algorithm and then followed by job scheduling & CloudSim & Time, bandwidth 
\\ \hline


\end{tabular}
}
\end{sidewaystable*}

\begin{sidewaystable*}[!htbp]
\caption{Comparison of Reliability-based Scheduling Algorithms}\label{T:h}
\hspace{8pt}
\centering
\scriptsize
\resizebox{\textwidth}{!}{%
\begin{tabular}{|g{2cm} p{2cm} p{2cm} p{2cm} p{2cm} p{2cm} p{2cm} p{1.9cm}|}
  \hline
   \textbf{Classification} & \textbf{Author} & \textbf{Basis} & \textbf{Advantages} & \textbf{Disadvantages} & \textbf{scheduling techniques} & \textbf{Experimental Scale} &\textbf{Experimental Parameters} 
\\ \hline

\makecell{Compute\\ resource-based,\\ Job-based,\\ Reliability-based} & Malik et al.~\cite{malik2012reliability} & Reliability assessment mechanism for scheduling resources & Reliability assessment algorithms for general applications and real time applications  & No security and energy parameters consideration & Max -min & Amazon EC2 cloud & Fault tolerance, time
\\ \hline

\makecell{Compute\\ resource-based,\\ Job-based,\\ Reliability-based} & Jing et al.~\cite{jing2015reliability} &  Model for fault-tolerant aware scheduling &  Low complexity  & No cost, time optimization & Adaptive secure scheduling algorithm & Simulated environment & Reliability
\\ \hline
  
\makecell{Compute\\ resource-based,\\ Task-based,\\ Reliability-based} & Abdulhamid et al.~\cite{latiff2016fault} & Uncountable numeric nodes for resource in clouds & Lower makespan & No optimization & League championship algorithm & CloudSim & Failure ratio, the failure slowdown and the performance improvement rate
\\ \hline

\makecell{Energy-aware ,\\ Energy source,\\ Reliability-based} & Tang et al.~\cite{tang2016energy} &  Reliability and energy-aware task scheduling architecture & To get good trade off among performance, reliability, and energy consumption & No support for cost optimization& Heuristic method & Discrete event simulation environment  & Schedule length, Energy consumption,  Application reliability 
 \\ \hline


\end{tabular}
}
\end{sidewaystable*}

\newcolumntype{g}{>{\columncolor{Gray}}p}
\begin{table*}[!htbp]
\caption{Comparison of Security based Scheduling Algorithms}
     \label{T:i}
\centering
\scriptsize
\resizebox{\textwidth}{!}{%
\begin{tabular}{|g{2cm} p{2cm} p{2cm} p{2cm} p{2cm} p{2cm} p{2cm} p{1.9cm}|}
  \hline
  \textbf{Taxonomy Classification} & \textbf{Author} & \textbf{Basis} & \textbf{Advantages} & \textbf{Disadvantages} & \textbf{scheduling techniques} & \textbf{Experimental Scale} & \textbf{Experimental Parameters} 

\\ \hline

Compute-resource and Data based  & Chejerla et al.~\cite{chejerla2017qos} &  Scheduling of resources in cloud integrated Cyber-physical Systems & Consideration of security, time & High complexity &Heuristic algorithm & Simulated environment & Speed up, resource utilization, makespan 
\\ \hline

Compute-resource based, VM based & Shetty et al.~\cite{shetty2016security}& VM placement techniques to reduce security risks & Reduced computing costs and deployment costs& No optimization criteria & Heuristic algorithm& Simulated environment & Cost, security risks & 
\\ \hline 

Compute resource based-workflow & Zeng et al.~\cite{zeng2015saba} & Scheduling algorithm for resource utilization & Low complexity & Ignored energy consumption & Clustering and prioritization algorithm  &  Simulated environment & Makespan and speed up 
\\ \hline 

Compute-resource based, VM based & Kashyap et al.~\cite{kashyap2014security}& Secure aware scheduling of real time based applications & Improved response time and overall security & High complexity & Priority Algorithm & Hypervisor & Deadline, Security 
\\ \hline 

Compute-resource based, Workflow & Liu et al.~\cite{liu2013ccbke} &  Scheme for security aware scheduling &  Reduced the computational load and execution time  & No cost optimization involved & Adaptive secure scheduling algorithm & KVM hyper-visor & Time unit consumed per computational load 
\\ \hline

Compute-resource based, Task based & Wang et al.~\cite{wang2012cloud} & Uncountable numeric nodes for resource in clouds& Provided scheduling of resources in secure way & Ignored cost & Bayesian  algorithm& CloudSim & Trust value, average schedule length 
\\ \hline

Compute-resource based, VM based & Afoulki et al.~\cite{afoulki2011security}& Security risk management in a cloud &  Less complexity & Consolidation issues while implementing policies & Greedy Algorithm & Simulated environment & VM placement time  
\\ \hline 

Compute-resource based, VM based & Bilogrevic et al.~\cite{bilogrevic2011meetings} & Scheduling services on the cloud for mobile devices & Enhanced Performance & No support cost optimization, Ignores power consumption by the network & Privacy aware scheduling schema & Testbed & Time, Data exchanged, privacy in approach 
 \\ \hline
\end{tabular}
}
%\tiny}

\end{table*}

\begin{sidewaystable*}[!htbp]
\caption{Comparison of Heuristic-based Scheduling Algorithms}\label{T:j}
\centering
\scriptsize
\resizebox{\textwidth}{!}{%
\begin{tabular}{|g{2cm} p{2cm} p{2cm} p{2cm} p{2cm} p{2cm} p{2cm} p{2cm}|}
  \hline
  % after \\: \hline or \cline{col1-col2} \cline{col3-col4} ...
 \textbf{Classification} & \textbf{Author} & \textbf{Basis} & \textbf{Advantages} & \textbf{Disadvantages} & \textbf{scheduling techniques} & \textbf{Experimental Scale} &\textbf{Experimental Parameters} 
   

\\ \hline

Compute resource based, job based & Gasior et al.~\cite{gkasior2016metaheuristic} & Parallel and distributed scheme  for  scheduling  jobs  & Multi-objective optimization, consideration of security risks also& No cost consideration & Genetic algorithm & Simulation Testbed & Flow time, makespan, turnaround time 
\\ \hline


Compute resource based, workflow based & Bousselm et al.~\cite{bousselmi2016qos} & QoS based & Consideration of QoS parameters & High complexity & Parallel Cat Swarm Optimization& Simulated environment  & Execution time, execution and storage cost, availability of resources and data transmission time 
\\ \hline

Compute resource based, job based & Cristian et al.~\cite{mateos2013aco} & Scheduler for job scheduling, consider static cloud  & Minimize weighted flowtime and makespan & does not handle energy consumption & Ant colony optimization and swarm intelligence approach & CloudSim & makepan 
\\ \hline

Compute resource based, workflow & Abrishami et al.~\cite{abrishami2013deadline} & Cost-optimized,  deadline-constrained execution of workflows in cloud. considered required run-time and data estimates in order to optimize workflow execution & Minimize execution cost with in deadline & Ignored data transfer time, security & PCP algorithm & Simulated environment & Normalized cost 
\\ \hline

Compute resource based, task based & Sen et al.~\cite{su2013cost} & Cost-efficient task-scheduling algorithm using two heuristic strategies & Reduced monetary costs & Ignored security & Heuristic strategies & Numerical experiments & Makespan 
\\ \hline

 Compute-resource based, Job based & Gutierrez-Garcia et al.~\cite{gutierrez2013family} & Scheduling of Bag-of-tasks based on allocation times of virtualized cloud resources & Makespan & Ignored cost & Heuristic algorithm &  Testbed & Makepan, overhead time
\\ \hline

Compute resource based, task based, energy based & Babu~\cite{ld2013honey} & Based  priority of tasks, designed load balancing algorithm & Maximize throughput & High operational complexity & Honey Bee algorithm & CloudSim & Makespan, Number of task migrations 
\\ \hline

Compute resource based, task based & Kousiouris et al.~\cite{kousiouris2011effects}& Virtual machines affect the performance of other VMs executing on the same node & Reduce performance overhead & Lacks implementation on a real-world cloud platform  & Genetic algorithm  &  Simulated environment & Degradation, test score delay 
\\ \hline

Compute resource based, Job based & Mezmaz et al.~\cite{mezmaz2011parallel} & Addressed the precedence-constrained parallel applications for cloud computing & Reduced energy consumption & High complexity of implementation and operation& Genetic algorithm & Simulated environment & Energy, speed up 
\\ \hline


Compute resource based, VM based & Torabzadeh et al.~\cite{torabzadeh2010cloud} & Flowshow job problem  & Minimized makespan and mean completion time & Not considered cost & Simulated annealing & Simulated environment & Computation time 
\\ \hline


\end{tabular}
}

\end{sidewaystable*}

\begin{sidewaystable*}[!htbp]
\caption{Comparison of ML-based Scheduling Algorithms}\label{T:j1}
 \hspace{8pt}
\centering
\scriptsize
\resizebox{\textwidth}{!}{%
\begin{tabular}{|g{2cm} p{2cm} p{2cm} p{2cm} p{2cm} p{2cm} p{2cm} p{2cm}|}
  \hline
  % after \\: \hline or \cline{col1-col2} \cline{col3-col4} ...
 \textbf{Classification} & \textbf{Author} & \textbf{Basis} & \textbf{Advantages} & \textbf{Disadvantages} & \textbf{scheduling techniques} & \textbf{Experimental Scale} &\textbf{Experimental Parameters}


\\ \hline

\makecell{Compute \\resource-based,\\ Job-based,\\ ML-based} & Mingxi Cheng et al.~\cite{cheng2018drl} &  Two stage resource provisioning and/or task scheduling processor & Consideration of energy consumption aspects & No security consideration  & Reinforcement approach & Simulation Testbed & Energy cost and runtime
\\ \hline


\makecell{Compute \\resource-based,\\ Workflow-based,\\ ML-based} & Hongzi Mao et al. ~\cite{mao2018learning} & Presented Decima scheduler & Consideration of low latency scheduling decisions & High complexity & Reinforcement learning method & Simulated environment  & Execution time
\\ \hline

\makecell{Compute \\resource-based,\\ Job-based,\\ ML-based} & Qingchen Zhang et al. ~\cite{zhang2017energy} & A hybrid dynamic voltage and frequency scaling (DVFS) scheduling algorithm   & Minimize energy consumption & does not handle security & Deep Q-learning model & Simulation environment & energy consumption
\\ \hline

\makecell{Compute \\resource-based,\\ Workflow-based,\\ ML-based} & Yuandou Wang et al. ~\cite{wang2019multi} & Focused on workflow scheduling & Minimize energy consumption & Ignored time, security & Markov method & Simulated environment & Energy efficiency
\\ \hline

\makecell{Compute \\resource-based,\\ Task-based,\\ ML-based} & Enda Barret et al. ~\cite{barrett2013applying} & Guided  an optimal decision in the process of resource allocation & Reduced Time & Ignored security & Deep Q learning method & Simulation & Makespan
\\ \hline


\end{tabular}
}

\end{sidewaystable*}

\begin{sidewaystable*}[p]

  % \centering
\caption{Comparison of Different Batch Resource Management Systems}\label{T:l}
\hspace{8pt}
\centering
\scriptsize
\resizebox{\textwidth}{!}{%
\begin{tabular}{|g{2cm} p{1.9cm} p{2.3cm} p{2cm} p{1.5cm} p{2cm}|}
  \hline
 \textbf{Classifaction} &   
 \textbf{Framework} & \textbf{Batch Features} &  \textbf{Cloud Bursting} & \textbf{Containers} & \textbf{Comment}  
\\ \hline

\makecell{Cloud bursting,\\ Cluster model} & Slurm~\cite{www-slurm} & 
Policy driven, backfill, exclusive and non-exclusive access to compute nodes & 
AWS, Azure, Google, Oracle & 
Yes & 
Open Source, popular
  \\ \hline

\makecell{Cloud bursting,\\ Cluster model} & Univa Grid Engine~\cite{www-univa} & 
Policy driven, backfill, exclusive and non-exclusive access to compute nodes, fault tolerant master & 
AWS, Azure, Google & 
Yes & 
previously SUN Grid Engine, Genias Codine
\\ \hline  
  
\makecell{Cloud bursting,\\ Cluster model} & Load Sharing Facility (LSF)~\cite{www-lsf} & 
Policy driven, backfill, exclusive, non-exclusive access to compute nodes &  IBM Cloud, AWS, Google and Azure & 
Yes & 
Previously OpenLava, IBM  Open Source 
\\ \hline

\makecell{Cloud bursting,\\ Cluster model} & Moab~\cite{www-moab} & 
Fairness policies, dynamic priorities, and extensive reservations &
AWS, Azure, Oracle, Google &
Yes & 
Open Source
\\ \hline
  
\makecell{Cloud bursting,\\ Cluster model} & Open Portable Batch System (OpenPBS)~\cite{Openpbs2018,Henderson1995} & 
Policy driven, backfill, exclusive and non-exclusive access to compute nodes, fault tolerant master & 
AWS, Azure, Google, Oracle & 
Yes & 
Open Source
\\ \hline



\end{tabular}
}

\end{sidewaystable*}

\begin{sidewaystable*}[!htbp]
\caption{Comparison of Different IaaS Models}\label{T:iaas}
\centering
\scriptsize
\resizebox{\textwidth}{!}{%
\begin{tabular}{|g{2cm} p{1.9cm} p{2cm} p{1.0cm} p{2cm} p{1.9cm} p{1.9cm} p{2cm}|}
  \hline

  \textbf{Classification} & 
\textbf{Provider or Framework} & \textbf{Pricing} &  
\textbf{Database RDS} & \textbf{Reliability} & \textbf{Monitoring} & 
\textbf{Base OS}   & \textbf{Programming Framework}
 \\ \hline
 
& Amazon EC2~\cite{AmazonEC22015} & Pay-as-you-go or Yearly, reserved, spot & My SQL, Ms SQL, Oracle & Good & Good & Linux and windows & Python, Java, PHP, Ruby 
\\ \hline

& Microsoft Window Azure~\cite{MicrosoftAzure2014} &  Pay-as-you-go, semester, year &  Microsoft SQL Database & Average & Average & Windows and linux & Java, Php, .net 
\\ \hline

& Rackspace~\cite{Rackspace2016} &  Pay-as-you-go  & MySQL & Good & Extensive  & Ubuntu & Java, Python 
\\ \hline

& Google App Engine~\cite{GoogleAppEngine2018} & Pay as you go & Cloud SQL & Extensive & good & linux, free BSD, windows & Python,  Java,  PHP and Go, Node.js
\\ \hline

& Cloud Sigma~\cite{CloudSigma2016} & Pay-as-you-go  & SQL & Good& Good & Average &Python, Java, PHP, Python, Ruby, Clojour 
\\ \hline

%Openstack~\cite{Openstack2018} & Pay-as you go, monthly  & My SQL& Good & Extensive & Linux, windows & Python, Perl, PHP,  
%\\ \hline
%\textbf{Digital Occean} & Pay-as-you-go or Monthly, semester, year  &100\%  & 5 & Poor & None & Python, Perl, PHP 
%\\ \hline
%Open Nebula~\cite{OpenNebula2018} &Subscription &My SQL  & High & Good & Linux & C,C++, Ruby, java, 
%\\ \hline

& Future Grid (discontinued)~\cite{las12fg-bookchapter,fox2013futuregrid} & Free Academic &  User Choice &  Good & Good &  Linux & Openstack, OpenCirrus, Eucalyptus, Cloudmesh
  \\ \hline
  
& Chameleon Cloud~\cite{las-chameleon} & Free Academic &  User Choice &  Moderate & Moderate &  Linux & Openstack
  \\ \hline
\end{tabular}
}
\end{sidewaystable*}

%\begin{sidewaystable*}[!htbp]
\begin{table*}[htb]
\caption{Comparison of Container-based Scheduling Algorithms}\label{T:z}
\hspace{8pt}
\centering
\scriptsize
\resizebox{\textwidth}{!}{%
\begin{tabular}{|g{2cm} p{2cm} p{2cm} p{2cm} p{2cm} p{2cm} p{2cm} p{2cm}|}
  \hline
 
\textbf{Classification} & \textbf{Author} & \textbf{Basis} & \textbf{Advantages} & \textbf{Disadvantages} & \textbf{scheduling techniques} & \textbf{Experimental Scale} &\textbf{Experimental Parameters}

\\ \hline

\makecell{Compute\\ resource-based,\\ Container-based} & Guerrero et al.~\cite{guerrero2018genetic} &   Optimize physical machine utilization & Increase resource utilization & High complexity & Genetic Algorithm & Simulation environment & Resource Utilization , Performance
\\ \hline

  \makecell{Compute\\ resource-based,\\ Energy-aware,\\ Container-based} & Hanaf et al.~\cite{hanafy2017novel} &  Container and host selection policies & Improved SLA & Highly complex & Pre-Selection method & Simulated environment & Energy Consumption
\\ \hline


\makecell{Compute\\ resource-based,\\ Container-based} & Medel et al.~\cite{medel2017client} & Scheduler for minimizing resource contentions & Reduce resource contention & Ignored Time optimization & Priority algorithm & Kubernetes &  Time
\\ \hline

\makecell{Compute\\ resource-based,\\ Container-based} & Dziurzanski et al. \cite{dziurzanskivalue} & Optimization of the container allocation & Easy to implement & Ignored network optimization & Heuristic method & Simulated environment & Performance
\\ \hline

\makecell{Compute\\ resource-based,\\ Container-based} & Guerrero et al. \cite{guerrero2018resource} & Optimized the deployment of micro services-based applications & Improved security & High complexity & Genetic algorithm & Simulation environment & Resource utilization
\\ \hline


\end{tabular}
}
%\tiny}

\end{table*}
%\end{sidewaystable*}


%\begin{sidewaystable*}[p]
\begin{table}[htb]
%\centering
\caption{Comparison of other Resource Management Systems}\label{T:distr-cloud}
\hspace{8pt}
\centering
\scriptsize
\resizebox{\textwidth}{!}{%
\begin{tabular}{|g{2cm} p{1.9cm} p{2cm} p{2cm} p{2cm} p{2cm} p{2cm} p{2cm}|}
  \hline
\textbf{Classification} &
  \textbf{Framework} & \textbf{Architecture} &  \textbf{Usage} & \textbf{Open source} & \textbf{Support} & \textbf{Applications}   & \textbf{Programming Framework} 
\\ \hline

\makecell{Datacenter-based} & Eagle~\cite{delgado2016job} & Hybrid & Differentiates short and long jobs & EPFL IC IINFCOM LABOS, Switzerland & Spark & Different workloads and Parallel jobs & Python, Java, PHP, Python, Ruby 
\\ \hline

\makecell{Cluster-based} & Hopper~\cite{ren2015hopper} &  Decentralized &  Speculation-aware job scheduler & Microsoft Research & Spark  & CPU intensive & Java, Php, .net
\\ \hline

\makecell{Cluster-based} & Tetris~\cite{grandl2015multi} & Centralized & Multi-resource bin-packing & Microsoft & Generic applications & CPU intensive & Python, Perl, Java, PHP, Ruby, Node.js, Erlang, Scala
\\ \hline

\makecell{PaaS-based,\\ Data-based} & Fawkes~\cite{ghit2014balanced} & Centralized & Dynamic resource balancing &TU Delft & Mapreduce frameworks & Data intensive & Python, Java, PHP, Python, Ruby 
\\ \hline

\makecell{Cluster-based} & Omega~\cite{schwarzkopf2013omega} &  Decentralized & Shared state abstraction & University of Cambridge & Custom applications & Parallel applications &Java, Php, .net
\\ \hline

\makecell{Peer-to-Peer} & OurGrid~\cite{andrade2003ourgrid} & Centralized  & Equitable Resource Sharing & Universidade Federal de CampinaGrand, Brazil & Generic applications  & Bag of Tasks  & Java, Python
\\ \hline

\makecell{PaaS-based,\\ Data-based} & Sparrow~\cite{ousterhout2013sparrow}& Decentralized & Randomized sampling approach &  U.C. Berkeley AMPLab & Spark& CPU intensive& Python, Perl, Java, PHP, Ruby, Node.js, Erlang, Scala, Clojure, .Net
\\ \hline

\makecell{PaaS-based,\\ Data-based} & Yarn~\cite{vavilapalli2013apache} & Monolithic  & Resource requests with containers &Hadoop  & Spark & Data intensive  &Python, Java, PHP, Python, Ruby, Clojour
\\ \hline

%Mesos~\cite{Mesos2018} & Two way protocol  & Pessimistic resource offers & University of California, Berkeley & Spark & CPU and Data Intensive & Python, Perl, PHP, Rest, Ruby, .net, C\#
%\\ \hline

\end{tabular}
}
\end{table}

%\end{sidewaystable*}





\end{document}
