\begin{abstract}

The growth and development of commercial and scientific applications in the cloud demand the creation of efficient resource management systems to coordinate the resources while addressing the heterogeneity of services, the inter-dependencies, and unpredictability of load posed by the users.
We present a resource scheduling taxonomy that originates from the experience of the authors in utilizing and managing multi-cloud environments. This study is backed up by a literature review that targets not only virtual machines but also container and Function as a Service frameworks. It justifies a proposed resource provider focused Y-cloud taxonomy and introduces an overview of existing scheduling techniques in cloud computing. As a result, this work can lead to a better understanding of the complex field of scheduling for clouds in general. Furthermore, the study promotes through the Y-cloud taxonomy, the vision of a layered scheduling architecture that will be useful for the implementation of application and resource-based scheduling frameworks in support of the NIST Big Data Reference Architecture.

\end{abstract}
